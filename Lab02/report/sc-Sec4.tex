\section{Unsharpen Mask}
\subsection{Defining and applying the Unsharpen Effect}
Applying this filter counterintuitively has the effect of sharpening the image. We first begin by passing the image $I$ through a low pass filter to get $I_l$. This can then be subtracted from $I$ to leave us with a new image that only contains the high frequency components in this image. We can then apply a gain of $\alpha$ to this image to control how much of the high frequency components we want to add back to the image. This then gives us our final sharpened image. If we raise $\alpha$ too much, we end up introducing sharpening artifacts to the image. However, if $\alpha$ is too small we don't end up with the desired sharpening effects. This is demonstrated in Figure \ref{fig:SharpenedImages}.

\begin{figure}[!h]
    \includegraphics[width=1\textwidth]{SharpenedImages.png}
    \centering
    \caption{Sharpening effect with 3 values of $\alpha$}
    \label{fig:SharpenedImages}
\end{figure}

\noindent Through trial and error I found that an $\alpha$ value of 4.7 provided good enough sharpening results without introducing too many artifacts. These artifacts are visible in the 3rd image as banding on the wall of the house and the fence.